\documentclass[12pt,a4paper]{article}
\usepackage[utf8]{inputenc}
\usepackage{amsmath}
\usepackage{amsfonts}
\usepackage{amssymb}
\usepackage{makeidx}
\usepackage{graphicx}
\usepackage[brazil]{babel}
\usepackage[pdftex,colorlinks=true,
citecolor=green,linkcolor=blue]{hyperref}

\author{Fernanda V. \emph{\&} Geferson L.}
\title{Minicurso: Introdução ao \LaTeX}
\usepackage{verbatim}
\begin{document}
\maketitle
\begin{abstract}
					\LaTeX\ é um editor de textos de alta qualidade tipográfica e de robusta formatação. 
					Vem sendo muito utilizado pela academia científica mas está se estendendo 
					a mais áreas do conhecimento. O minicurso tem o objetivo de introduzir 
					a linguagem \LaTeX\ , expondo sua estrutura básica, os 
					principais comandos utilizados, o ambiente matemático e como criar e  
					lidar com citações e referências bibliográficas. Também será mostrado como criar e 
					editar uma apresentação no ambiente beamer (análogo ao Power Point). 
					Além disso, o minicurso fornecerá um
					panorama geral de maneiras para se produzir documentos científicos, tais como artigos e livros, 
					no formato padrão das revistas de publicações.
					Espera-se que no final do 
					minicurso, os alunos tenham adquirido uma ideia geral do funcionamento do \LaTeX\ 
					e que estejam aptos a criar documentos e apresentações com fórmulas 
					matemáticas e controle de referências.
\end{abstract}
\section*{Instalando o \LaTeX}
\subsection*{Usuários Linux (recomendado)}
		 Para instalar o texlive-full, que são todos os pacotes necessários da base \TeX 
		 para compilar os documentos, abra o terminal e digite: 
		\begin{verbatim}
		sudo apt install texlive-full.
		\end{verbatim}
		Há três compiladores muito usados:  TexStudio (recomendado), Kile e  TexMaker. Para 
		instalar, digite no terminal
		\begin{verbatim}
		sudo apt install texstudio ou kile ou texmaker.
		\end{verbatim}
		O TexStudio e o TexMaker possuem um visualizador de pdf embutido! 
		
\subsection*{Usuários Windows}
Instalar o {\bf texlive-full} da base \TeX, vá em \url{https://www.tug.org/texlive/acquire-netinstall.html},
baixe e execute o arquivo ``{\bf install-tl-windows.exe}''. Siga as instruções de instalação para o 
pacote completo (maior que 4 Gb). Se tiver problemas na instalação, é possível baixar o arquivo .iso 
completo da base \TeX ($\sim 4$GB), extraia e execute o instalador. 

Após, baixe e instale o Texstudio \url{http://www.texstudio.org/}.

\section*{Dúvidas}
Qualquer dúvida, envie um e-mail em conjunto para \underline{gefersonlucatelli@gmail.com} 
e \underline{fervanucci@gmail.com}.
\end{document}